\begin{flushleft}
	\setcounter{page}{1}
	\section{Einleitung}
	In dieser Projektdokumentation präsentiert der Autor den Ablauf seines Abschlussprojekts, welches im Rahmen seiner Ausbildung zum Fachinformatiker für Anwendungsentwicklung durchgeführt wurde. Das Projekt fand bei der Gesellschaft für Systemtechnik, Softwareentwicklung und Datenverarbeitungsservice mbH (\acs{GSSD}) in Teltow statt. Aktuell beschäftigt die \acs{GSSD} vier Mitarbeiter und betreut als IT-Dienstleister zahlreiche kleine und mittlere Unternehmen.
	\\
	Darüber hinaus ist die (\acs{GSSD}) Partner verschiedener Anbieter von Warenwirtschaftssoftware und offeriert in diesem Bereich Anpassungs- und Individualisierungsdienstleistungen. Ein weiterer Geschäftszweig der \acs{GSSD} besteht im Verkauf von Hardware für Unternehmen und Privatkunden sowie in der dazugehörigen Installation und Wartung der Systeme.
	\\
	Das Hauptziel dieses Projekts bestand darin, die im Rahmen der Ausbildung erworbenen Kenntnisse und Fähigkeiten anzuwenden und eine webbasierte Anwendung zu entwickeln, die den Anforderungen der \acs{GSSD} gerecht wird. Die Projektdokumentation beschreibt den gesamten Ablauf, von der Planung und Analyse über die Implementierung bis hin zur Bewertung und Dokumentation der Ergebnisse.

	\subsection{Projektbeschreibung}
	Die \acs{GSSD} stellt für einen Kunden ein innovatives Überwachungssystem bereit, das auf dessen Servern laufende Anwendungen, welche als Dienste oder im Hintergrund agieren, effizient überwacht. Das bisherige System lieferte nur ungenügende Informationen, weshalb eine modernere Lösung entwickelt wird, die ein umfassendes Bild der Anwendungslandschaft ermöglicht.
	\\
	Das Herzstück des Projekts ist die Erstellung eines intuitiven Dashboards, das den Anwendungsstatus klar und ansprechend visualisiert. Das Überwachungssystem besteht aus einer leistungsstarken Backend- und einer benutzerfreundlichen Frontend-Anwendung, die es dem Administrator ermöglichen, alle laufenden Anwendungen und deren Interaktionen effektiv zu überwachen. Die Anwendungen selbst generieren kontinuierlich Status- und Aktivitätsdaten, die in Datenbanktabellen gespeichert werden. Das Backend verarbeitet diese Daten und bereitet sie für die Darstellung im Frontend auf.
	\\
	Mithilfe von Websockets wird eine nahtlose Kommunikation zwischen Frontend und Backend gewährleistet. Dadurch können Benutzer den Status der Anwendungen jederzeit auf dem Frontend einsehen, solange eine Verbindung besteht. Zudem erlaubt eine intelligente Filterfunktion im Backend, die angezeigten Informationen gezielt zu verfeinern.
	\subsection{Projektziel}
	Das Hauptziel des Überwachungssystems besteht darin, eine Echtzeit-Visualisierung des Anwendungsstatus auf einem ansprechenden Dashboard bereitzustellen, das Administratoren eine umfassende Überwachung der Anwendungen und deren Interaktionen ermöglicht. Die Anwendungen senden kontinuierlich Daten, die vom Backend verarbeitet und in Datenbanktabellen gespeichert werden. Die Kommunikation zwischen Frontend und Backend erfolgt reibungslos über Websockets.
	\\
	Benutzer können den Anwendungsstatus abrufen und neue Sensoren hinzufügen, während das System unter Verwendung zukunftsweisender Technologien wie HTML, VueJs, CSS, Bash, SQL und NodeJs entwickelt wird. Im Falle kritischer Fehler stellt ein automatisches Benachrichtigungssystem sicher, dass Administratoren umgehend informiert werden und angemessene Maßnahmen ergreifen können.
	\subsection{Projektbegründung}
	Eine effektive Überwachung von Anwendungen und Systemen ist im Bereich der Informationstechnologie von entscheidender Bedeutung, da sie eine frühzeitige Fehlererkennung ermöglicht und negative Auswirkungen auf Betrieb und die Verfügbarkeit minimiert. Ein zuverlässiges Überwachungssystem ermöglicht dem Administrator, die Leistung und den Zustand von Komponenten kontinuierlich zu überwachen, um schnell auf auftretende Probleme zu reagieren. Dadurch werden die Qualität und Verfügbarkeit der Anwendungen verbessert und die Kundenzufriedenheit erhöht.
	\\
	Die Implementierung eines solchen Systems stellt eine reibungslos funktionierende IT-Infrastruktur und eine hohe Kundenzufriedenheit sicher. Mit einem modernen und intuitiven Überwachungssystem können Administratoren potenzielle Probleme frühzeitig erkennen und beheben, bevor sie zu schwerwiegenden Störungen oder Ausfällen führen. Dies trägt dazu bei, den kontinuierlichen Betrieb der Anwendungen und Dienste zu gewährleisten und den Kunden einen zuverlässigen Service zu bieten.

	\subsection{Projektschnittstellen}
	Das Projekt umfasst folgende Schnittstellen:
	\begin{itemize}
	\item Geschäftsführung: Informationen über Projektfortschritte und Entscheidungen, die das Projekt beeinflussen, werden an die Geschäftsführung weitergegeben.
	\item IT-Abteilung: Die IT-Abteilung ist für die technische Umsetzung zuständig und arbeitet eng mit dem Projektteam zusammen.
	\item Externe Dienstleister: Schnittstellen müssen klar definiert werden, um eine reibungslose Zusammenarbeit mit externen Systemen und Dienstleistern zu gewährleisten.
	\item Benutzer: Die Einbindung und Information der Benutzer ist wichtig, um deren Anforderungen gerecht zu werden und eine hohe Akzeptanz des Systems zu erreichen.
	\item Datenbanken und Anwendungen: Die Kommunikation mit anderen Datenbanken und Anwendungen ist erforderlich, um Informationen effizient verarbeiten und austauschen zu können.
	\item Netzwerk- und Sicherheitsinfrastruktur: Schnittstellen zu Netzwerk- und Sicherheitsinfrastruktur müssen definiert werden, um mögliche Anpassungen oder Erweiterungen reibungslos durchführen zu können.
	\end{itemize}
	\subsection{Projektgrenzen}
	\begin{itemize}
	\item Der Bau von Sensoren ist nicht Teil dieses Projekts.
	\item Schulungen für Benutzer oder Mitarbeiter sind nicht Bestandteil des Projekts.
	\item Das Überwachungssystem wird lediglich von einem bestimmten Team oder einer Abteilung verwendet und nicht von der gesamten Belegschaft.
	\end{itemize}
	\subsection{Use-Case}
	\subsubsection{Der Prozessabläufe}
	\begin{itemize}
	\item Ablauf 1 beim Administrator:
		\begin{itemize}
		\item Administrator öffnet Dashboard.
		\item Dashboard zeigt Anwendungsübersicht und aktuellen Zustand.
		\item Administrator filtert Anwendungsliste nach Kriterien.
		\item Bewertung jeder Anwendung wird angezeigt.
		\item Bei kritischem Zustand oder Problem wird Administrator benachrichtigt.
		\item Administrator reagiert auf Benachrichtigung (Problem lösen, System normalisieren).
		\item Administrator passt Überwachungssystem an, beim Anpassen ist eine Anmeldung erforderlich (Bewertungsfaktoren ändern, Anwendungen hinzufügen oder entfernen).
		\end{itemize}
		\item Ablauf 2 beim Administrator:
		\begin{itemize}
		\item Administrator öffnet Benachrichtigungseinstellungen im Überwachungssystem.
				\item Administrator konfiguriert Benachrichtigungsschwellenwerte.
				\item Überwachungssystem speichert die Konfiguration und sendet automatisch Benachrichtigungen, wenn Schwellenwerte überschritten werden.
				\item Administrator erhält Benachrichtigungen und ergreift entsprechende Maßnahmen.
		\end{itemize}
		\item Ablauf beim Entwickler
		\begin{itemize}
		\item Entwickler öffnet Sensorintegrationsseite im Überwachungssystem.
		\item Entwickler gibt erforderliche Informationen für den neuen Sensor ein.
		\item Überwachungssystem validiert eingegebene Informationen.
		\item Bei erfolgreicher Validierung wird neuer Sensor dem System hinzugefügt und in der Sensorliste angezeigt.
		\item Bei fehlerhafter Validierung wird Entwickler benachrichtigt und kann die Eingaben korrigieren.
		\end{itemize}
		\end{itemize}
		\subsubsection{Anwendungsfälle}
		Resultierend aus den Prozessabläufe lassen sich die folgende Anwendungsfälle und Akteuren ableiten:
	\begin{itemize}
	\item Administratoren als Akteur (Administratoren überwachen Systemzustand und reagieren auf potenzielle Probleme):
	\begin{itemize}
	\item Anmelden im Überwachungssystem
	\item Öffnen von Dashboard
	\item Filtern Anwendungsliste nach Kriterien
	\item Konfigurieren des Überwachungssystems (Bewertungsfaktoren ändern, Anwendungen hinzufügen oder entfernen)
	\item Öffnen der Benachrichtigungseinstellungen im Überwachungssystem
	\item Konfigurieren Benachrichtigungsschwellenwerte
	\end{itemize}

		\item Entwickler als Akteur(integriert neue Sensoren ins System):
		\begin{itemize}
				\item  Öffnen von Sensorintegrationsseite im Überwachungssystem
				\item  Eingeben der erforderlichen Informationen für den neuen Sensor

		\end{itemize}
		\end{itemize}



	\section{Projektplanung}

	\subsection{Projektphasen}
	Zur Durchführung des Projektes standen 80 Stunden zur Verfügung. Für die Ausarbeitung des Konzeptes
	wird die meiste Zeit eingeplant. Eine ausführliche Projektplanung ist im voraus nötig, allerdings kann auf
	Grund fehlender Erfahrung keine genaue Zeit geschätzt werden, es sind 15 Stunden angesetzt. Da der
	Autor zum Zeitpunkt der Durchführung dieser Projektarbeit mit den anzubindenden Komponenten und
	Problemen bei der täglichen Arbeit in der \acs{GSSD} vertraut ist, wird ein Zeitaufwand von 6 Stunden für die
	Analysephase als realistisch eingeschätzt. Die weiteren Phasen werden als unkritisch eingestuft und
	sollten in der Ausarbeitung um maximal eine Stunde je Phase abweichen.
	\\
	Tabelle 1 zeigt die grobe Zeitplanung aus dem Projektantrag.


	\subsection{Ressourcenplanung}
	Anschließend wurden verwendete Ressourcen im Anhang A.2: Verwendete Ressourcen auf Seite ii
	aufgelistet, die während des Projekts eingesetzt wurden. Die Planung berücksichtigt
	sowohl Hard- als auch Software-Ressourcen sowie das beteiligte Personal.
	Um Kosten zu minimieren, wurde bei der Auswahl der verwendeten Software darauf geachtet,
	dass keine Lizenzgebühren anfallen, erforderliches Fachwissen vorhanden ist und die
	Architekturrichtlinien der \acs{GSSD} eingehalten werden.
	Die Architekturrichtlinien der \acs{GSSD} legen unter anderem die Nutzung von
	VSCode als Entwicklungsumgebung und den Jenkins-Server als Werkzeug für die Continuous
	Integration (CI) fest. Durch die Einhaltung dieser Richtlinien wird sichergestellt,
	dass das Projekt den Anforderungen der \acs{GSSD} entspricht und nahtlos in die bestehende
	Infrastruktur integriert werden kann.
%\ref{appendix:a0}%
%ac{}%
	\subsection{Entwicklungsprozess}
	In der \acs{GSSD} erfolgt die Entwicklung von Kundenprojekten üblicherweise durch den Einsatz der agilen Scrum-Methode. Der Entwicklungsprozess gliedert sich in mehrere ein- oder zweiwöchige Sprints, wobei Github zur Verwaltung der Aufgaben genutzt wird. Die Aufgaben werden den Sprints zugeordnet und während der wöchentlichen Sprintreviews überprüft.\\
Die umgesetzten Aufgaben werden gemeinsam mit dem Projektleiter kontrolliert, getestet und analysiert, um ein effektives Feedback und kontinuierliche Verbesserungen während des gesamten Entwicklungsprozesses zu gewährleisten. Diese strukturierte und kollaborative Vorgehensweise stellt sicher, dass die Endprodukt den Anforderungen und Erwartungen entsprechen.\\

	\section{Analysephase}
	\subsection{Ist-Analyse}
	Der Kunde der \acs{GSSD} verwendet auf seinen Servern verschiedene Anwendungen als Dienste oder im Hintergrund. Zur Überwachung dieser Anwendungen hat der Kunde bereits ein Überwachungssystem im Einsatz. Allerdings erweist sich dieses System als äußerst unzureichend und ineffizient in der Darstellung der Zustände und Informationen, die für die effektive Überwachung der Anwendungen erforderlich sind.
	\\
	Die Ist-Analyse ergab folgende gravierende Schwachstellen im aktuellen Überwachungssystem des Kunden:
	\begin{itemize}
	\item stark eingeschränkte und unvollständige Anzeige von Systemzuständen
	\item mangelhafte Integration und Erkennung von Sensoren, was zu Informationslücken führt
	\item fehlende oder stark verzögerte Benachrichtigungen bei Systemproblemen, wodurch das Eingreifen des Administrators erschwert wird
	\item Schwierigkeiten bei der Skalierbarkeit und Anpassungsfähigkeit an neue Anforderungen, was die Erweiterung und Anpassung des Systems behindert
	\item unübersichtliche und verwirrende Benutzeroberfläche, die eine effiziente Nutzung erschwert
	\item hohe Latenzzeiten und Leistungseinbußen bei der Überwachung von Anwendungen
	\end{itemize}

	Um die Zuverlässigkeit und Funktionalität der überwachten Anwendungen sicherzustellen, müssen diese Schwachstellen dringend angegangen werden. Die Ergebnisse der Ist-Analyse bilden die Grundlage für die anschließende Planung und Implementierung von Verbesserungsmaßnahmen für das Überwachungssystem.
	\subsection{Workflow}

	Der Code definiert sowohl ein Datenmodell als auch einen Sensor mit verschiedenen Eigenschaften wie Name, Typ, Datentyp, Einheit, Minimum- und Maximumwert, Status, Zeitstempel und weiteren Optionen.\\
	Es gibt zwei Arten von Sensoren: echte Sensoren und virtuelle Sensoren. Echte Sensoren senden kontinuierlich Daten an das Backend, während virtuelle Sensoren SQL-Abfragen an das Backend senden, um Daten aus der Datenbank abzurufen.
	\\
	\begin{enumerate}
		\item Der Prozess der Sensorregistrierung beginnt damit, dass der Sensorbauer eine API-Anfrage an das Backend sendet und seine Anmeldeinformationen übermittelt. Das Backend prüft, ob der Sensor bereits registriert ist. Wenn nicht, wird der Sensor in der Datenbank registriert und die Anmeldeinformationen werden an den Sensorbauer zurückgesendet.
		\item Nach Erhalt der ID des Sensor kann der Sensor regelmäßig Daten an das Backend übermitteln und Konfigurationsdaten abrufen. Das Backend ruft die Konfigurationsdaten des Sensors über die entsprechende ID aus der Datenbank ab und aktualisiert seine Eigenschaften, wie Name, Typ, Datentyp und Einheit.
		\item Der Sensor kann über die show-Eigenschaft angeben, ob er seine Daten an das Frontend senden und über saveData, ob er seine Daten in der Datenbank speichern möchte. Wenn der Sensor Daten senden möchte, aktualisiert er das Datenfeld im Sensor-Modell und sendet es an das Backend.
		\item Das Backend prüft, ob die Daten gemäß der in der variablen DataRetentionPeriodInMonths festgelegten Aufbewahrungsfrist gespeichert werden sollen. Wenn der saveData-Wert auf true gesetzt ist, speichert das Backend die Daten in der Datenbank und aktualisiert den Sensor-Status und den Zeitstempel.
		\item Wenn der Sensor ein virtueller Sensor ist, kann er über die commendsql-Eigenschaft SQL-Abfragen an das Backend senden, um Daten aus der Datenbank abzurufen. Das Backend führt die Abfrage aus und sendet das Ergebnis an den virtuellen Sensor.
		\item Das Frontend empfängt die Daten von den aktiven Sensoren und aktualisiert die Benutzeroberfläche entsprechend. Das Backend überwacht den Status der Sensoren und informiert den Sensorbauer über Probleme, die auftreten könnten, wie z.B. Sensorausfälle oder Datenbankfehler.
		\item Der Sensorbauer kann über die API auch neue Sensoren registrieren, bestehende Sensoren aktualisieren oder löschen sowie weitere Konfigurationen durchführen. Das Backend stellt hierfür entsprechende Funktionen bereit.
	\end{enumerate}



	\subsection{Soll-Konzept}
	In Zusammenarbeit mit Vorgesetzten wurde das folgende Soll-Konzept entwickelt, welches die vom Kunden gestellten Anforderungen an das Überwachungssystem beschreibt.
	Das Überwachungssystem soll alle relevanten Anwendungen auf den Servern des Kunden kontinuierlich überprüfen und bei Fehlererkennung oder Ausfällen den Administrator zuverlässig benachrichtigen. Dabei legt der Kunde Wert auf eine einfache Konfiguration und Verwaltung des Systems.
	Um sicherzustellen, dass das Überwachungssystem den Erwartungen des Kunden gerecht wird, müssen zusätzlich bestimmte Leistungskriterien festgelegt werden. Dazu zählen beispielsweise die maximale Anzahl der zu überwachenden Anwendungen, die Reaktionszeit bei Fehlermeldungen und die Verfügbarkeit des Überwachungssystems.

	\subsection{Analyse möglicher Probleme}

	Mögliche Probleme, die während der Projektumsetzung auftreten könnten, sind:
	\begin{enumerate}
	\item Integration mit vorhandenen Systemen und Anwendungen,
	\item Datenschutz und Sicherheit,
	\item Skalierbarkeit und Anpassungsfähigkeit,
	\item Kompetenz der beteiligten Mitarbeiter,
	\item Budget- und Zeitbeschränkungen.
	\end{enumerate}
	\subsubsection{Analyse Fremdsoftware/Komponenten}
	Diese Probleme sollten frühzeitig im Projektverlauf berücksichtigt und geeignete Lösungsansätze entwickelt werden. Enge Zusammenarbeit und Kommunikation zwischen allen Beteiligten sind entscheidend.
	\subsection{„Make or Buy“-Entscheidung}

	Die “Make or Buy”-Entscheidung ist eine wichtige Entscheidung, die Unternehmen treffen müssen, wenn sie sich entscheiden, ob sie eine Komponente oder einen Prozess intern entwickeln („Make“) oder von externen Anbietern kaufen („Buy“) sollen. Bei der Entscheidung, ob ein maßgeschneidertes Überwachungssystem intern entwickelt oder eine vorhandene Lösung wie Checkmk gekauft werden soll, sind mehrere Faktoren zu berücksichtigen:

	\begin{itemize}
	\item \textbf{Kosten:} Die Entwicklung eines maßgeschneiderten Überwachungssystems kann hohe Kosten verursachen, sowohl in Bezug auf Entwicklungszeit als auch auf Ressourcen. Allerdings können die langfristigen Kosteneinsparungen und die verbesserte Effizienz die anfänglichen Kosten rechtfertigen. Bei einer Kaufentscheidung können die Kosten für Lizenzen und Support ebenfalls ins Gewicht fallen.

	\item \textbf{Zeit:} Die Entwicklung eines eigenen Überwachungssystems kann viel Zeit in Anspruch nehmen. Der Kauf einer fertigen Lösung ermöglicht eine schnellere Implementierung, wobei jedoch Anpassungen und Integrationen möglicherweise zusätzliche Zeit beanspruchen.

	\item \textbf{Expertise:} Die Verfügbarkeit von internen Ressourcen und das Know-how spielen eine wichtige Rolle bei der Entscheidungsfindung. ich als interner Mitarbeiter verfüge über fundierte Kenntnisse in der Entwicklung solcher Systeme, kann die Entwicklung eines maßgeschneiderten Überwachungssystems erfolgreich bewältigen.

	\item \textbf{Anpassungsfähigkeit und Flexibilität:} Eine Eigenentwicklung ermöglicht größere Anpassungsfähigkeit und Flexibilität, um auf die spezifischen Bedürfnisse und Anforderungen des Unternehmens einzugehen.

	\item \textbf{Langfristige Perspektive:} Bei der Entscheidungsfindung sollte die langfristige Perspektive berücksichtigt werden. Eine Eigenentwicklung kann langfristig bessere Kontrolle, Anpassungsfähigkeit und mögliche Kosteneinsparungen bieten, während der Kauf einer Lösung kurzfristige Vorteile in Bezug auf Implementierungsgeschwindigkeit und Support bieten kann.


	\end{itemize}

	Nach Abwägung der oben genannten Faktoren und unter Berücksichtigung der Schwächen von Checkmk sowie der Vorteile einer Eigenentwicklung könnte der Kunde zu dem Schluss kommen, dass eine maßgeschneiderte Überwachungslösung langfristig bessere Ergebnisse und eine effektivere Überwachung der Anwendungen ermöglicht. In diesem Fall wäre die “Make”-Entscheidung für die Entwicklung eines eigenen Überwachungssystems die bevorzugte Wahl.

	Der Fokus des Projektes liegt auf der Eigenentwicklung einer maßgeschneiderten Softwarelösung, die den spezifischen Anforderungen und Bedürfnissen des Kunden entspricht. Die Vorteile einer Eigenentwicklung liegen in der Möglichkeit, Workflow und Benutzerfreundlichkeit gezielt zu gestalten und eine vollständige Kontrolle über den Code und dessen Implementierung zu haben. Dies ermöglicht eine effektive Fehlerbehebung und die Implementierung neuer Funktionalitäten. Ein weiterer Vorteil besteht darin, dass die Software kontinuierlich verbessert und an die sich ändernden Anforderungen des Anwenders angepasst werden kann. Daher ist eine Eigenentwicklung die ideale Lösung für dieses Projekt.

	\subsubsection{Projektkosten}
	Im Folgenden werden die Kosten für das Projekt detailliert aufgeschlüsselt. Neben den Personalkosten für Entwickler und weitere Projektbeteiligte sind auch die Kosten für die in Abschnitt 2.2 (Ressourcenplanung) aufgeführten Ressourcen einzuplanen.
	\\
	Da die genauen Personalkosten vertraulich sind, basiert die Kalkulation auf Stundensätzen, die von der Personalabteilung festgelegt wurden. Diese Stundensätze beinhalten hauptsächlich das Bruttogehalt und die Sozialabgaben des Arbeitgebers. Zusätzlich fallen Kosten für die Nutzung der Ressourcen an.
	\\
	Zusätzlich sollen die Kosten fürs Büroräume, Stromkosten,
	Kommunikationskosten und so weiter im Betracht gezogen.
	Aus Gründlen des Datenschutzes sind keine genauen Zahlen verfügbar.
	Deswegen werden 300,00€ pauschal berechnet.
	Für Mitarbeiter wird ein Stundensatz von 35,00 € angesetzt,
	während der Stundensatz für Auszubildende bei 10,50 € liegt.
	Für die Ressourcennutzung werden pauschal 15,00 € veranschlagt.
	\\
	Die Durchführungszeit des Projektes beträgt 80 Stunden. In Tabelle 1: Kostenaufstellung sind die
	Kosten unterteilt nach den einzelnen Projektvorgängen aufgelistet, sowie summiert dargestellt, um die
	Gesamtkosten, die während des Projektes anfallen, zu erhalten. Diese belaufen sich auf 1330,00€.

	\begin{table}[h]
		\centering
		\begin{tabular}{ >{\bfseries}l l r r r }
			\rowcolor[HTML]{127017}
		\textbf{\color{white}Vorgang} & \textbf{\color{white}Mitarbeiter} & \textbf{\color{white}Zeit} & \textbf{\color{white}kosten pro Stunde} & \textbf{\color{white}Gesamt} \\
		Entwicklungskosten im Rahmen der GMS & 1x Auszubildende & 80 & 10,50 & 840,00 \\
		\rowcolor[HTML]{e1efd9}
		Abnahme der Dokumentation & 1x Mitarbeiter & 2h & 35,00€/h & 70,00€ \\
		Aufsicht bei Projektplanung & 1x Mitarbeiter & 1h & 35,00€/h & 35,00€ \\
		\rowcolor[HTML]{e1efd9}
		Hilfestellung bei Problemen & 1x Mitarbeiter & 2h & 35,00€/h & 70,00€ \\
		Pauschalkosten &  &  &  & 315,00€ \\
		\hline
		\rowcolor[HTML]{127017}
		\multicolumn{4}{r}{\textbf{\color{white}Gesamt}} & \textbf{\color{white}1330,00 €} \\
		\end{tabular}
		\caption{Kostenübersicht}
		\label{tab:kostenuebersicht}
		\end{table}
	\subsubsection{Amortisationsdauer}
	Die Ermittlung der Amortisationsdauer für den Kunden in diesem Projekt gestaltet sich äußerst komplex, da es weder Vergleichsdaten aufgrund fehlender ähnlicher Anwendungen gibt, noch konkrete Zeitaufwandsangaben vorliegen, wie bereits in Abschnitt 3.2 erwähnt.

	Für die \acs{GSSD} hingegen amortisiert sich das Projekt relativ schnell, da die im Vertrag mit dem Kunden vereinbarten Personentage monatlich in Rechnung gestellt und beglichen werden. Zusätzliche Kosten, die nach der Inbetriebnahme anfallen, wie beispielsweise die Anmietung von Servern zur Bereitstellung der produktiven Anwendung, werden direkt vom Kunden getragen. Daher entstehen für die \acs{GSSD} keine weiteren Kosten, die nicht durch die Zahlungen des Kunden gedeckt wären.
	\subsection{Lastenheft}
	Für die Überwachung eines komplexen Systems gibt es verschiedene Ansätze. Ein möglicher Ansatz ist die Erstellung einer Gesamtbewertungsanzeige, die den Zustand des gesamten Systems anzeigt. Dazu können folgende Schritte durchgeführt werden:

	\begin{enumerate}
	\item Identifikation der relevanten Komponenten und Sensoren,
	\item Definition von Schwellwerten für jeden Sensor,
	\item Zusammenfassung der Sensorwerte in einer geeigneten Form,
	\item Definition von Alarmstufen für jeden Schwellwert,
	\item Integration von Bewertungsregeln,
	\item Anzeige der Bewertung auf eine einfache und intuitive Weise,
	\item Automatische Alarmierung bei kritischen Zuständen des Systems,
	\item Langzeitanalyse der Sensorwerte für Trends und Muster im Systemverhalten,
	\item interaktive Dashboards für detaillierte Ansichten und individuelle Einstellungen.
	\end{enumerate}



	\section{Entwurfsphase}

	\subsection{Zielplattform}
	Die zu entwickelnde Lösung ist eine Web-Applikation.
	Das System besteht aus Backend und Frontend, welche über Websockets Daten austauschen.
	Die Datenbank wurde mit mariaDB realisiert und enthält Tabellen für Anwendungsdaten.
	Das Backend (Node.js, Express) liest und bewertet die Daten und das Frontend (Vue.js) zeigt auf einem Dashboard an.
	\subsection{Architekturdesign}
	Das Architekturdesign für die Anwendungsüberwachung besteht aus mehreren Komponenten, die gemeinsam eine umfassende Monitoring-Lösung bieten. Die Architektur beinhaltet das Backend, das Frontend und das Überwachungssystem selbst.
	\\
	Das Backend wird in Node.js entwickelt und ist für die Verwaltung der Datenbank und der API-Schnittstellen verantwortlich. Es sammelt und analysiert Daten von verschiedenen Anwendungen und speichert sie in der Datenbank. Das Backend verwendet eine REST-API, um die Daten an das Frontend weiterzugeben.
	\\
	Das Frontend, entwickelt in Vue.js, stellt eine grafische Benutzeroberfläche bereit, die die Anwendungsüberwachung ermöglicht. Es greift auf die API-Schnittstellen des Backends zu, um Daten anzuzeigen und interaktive Funktionen wie das Hinzufügen oder Entfernen von Anwendungen oder das Einstellen von Warnmeldungen bereitzustellen.
	\\
	Das Überwachungssystem setzt sich aus Sensoren zusammen, die in jeder Anwendung integriert sind und Daten über deren Leistung und Verfügbarkeit sammeln. Diese Sensoren senden in regelmäßigen Abständen Daten an das Backend, um analysiert und gespeichert zu werden. Bei Problemen können die Sensoren automatisch Warnmeldungen auslösen.
	\subsection{Entwurf der Benutzeroberfläche}
	Der Entwurf der Benutzeroberfläche für das Überwachungssystem besteht aus verschiedenen Elementen, um eine intuitive und ansprechende Umgebung für die Benutzer zu schaffen. Hier sind die Hauptkomponenten des Benutzeroberflächenentwurfs:
	\begin{enumerate}
	\item Navigation: Eine Seitenleiste oder ein Menü am seitlichen Rand der Benutzeroberfläche ermöglichen den Benutzern den Zugriff auf verschiedene Abschnitte der Anwendung, z. B. Dashboard, Anwendungsverwaltung und Einstellungen.
	\item Dashboard: Das Dashboard ist der zentrale Bereich der Benutzeroberfläche, in dem die aktuelle Statusinformationen der überwachten Anwendungen angezeigt werden. Es kann Kacheln oder Karten enthalten, die für jede Anwendung einen schnellen Überblick über den Zustand, die Leistung und etwaige Warnmeldungen bieten.
	\item Anwendungsverwaltung: In diesem Abschnitt können Benutzer neue Anwendungen hinzufügen, vorhandene Anwendungen entfernen oder konfigurieren und die Sensoren für die Überwachung anpassen.
	\item Einstellungen: Das ist ein Bereich für die Verwaltung von Benutzerkonten, Systemeinstellungen und Benachrichtigungsoptionen.
	\item Filter- und Suchfunktion: Sie bitten eine Möglichkeit für Benutzer, die angezeigten Anwendungen und Daten schnell zu filtern oder nach bestimmten Anwendungen oder Kriterien zu suchen.
	\item Warnmeldungen: Eine Liste oder ein Bereich, der die aktiven Warnmeldungen und kritischen Ereignisse für die überwachten Anwendungen anzeigt. Benutzer können die Warnmeldungen ein- oder ausblenden und Details zu jedem Ereignis anzeigen.
	\item Diagramme und Statistiken: Für jede Anwendung können detaillierte Diagramme und Statistiken zur Leistung und Verfügbarkeit angezeigt werden. Dies kann in Form von Liniendiagrammen, Balkendiagrammen oder Tortendiagrammen erfolgen, je nach Art der Daten und der gewünschten Darstellung.
	\item Responsives Design: Die Benutzeroberfläche sollte so gestaltet sein, dass sie auf verschiedenen Bildschirmgrößen und Geräten gut aussieht und funktioniert, einschließlich Desktop-Computern, Tablets und Mobiltelefonen.
	\end{enumerate}
	Durch die Kombination dieser Elemente wird eine benutzerfreundliche und effektive Benutzeroberfläche geschaffen, die es den Benutzern ermöglicht, die Überwachungsinformationen für ihre Anwendungen leicht zu überprüfen und zu verwalten.

	\subsection{Datenmodell}
	Basierend auf dem Vertrag zwischen der \acs{GSSD} und dem Kunden sowie den zu berücksichtigenden Anwendungsfällen wurde vom Autor die erforderliche Datenbankstruktur analysiert und mithilfe von Node.js-Migrationen erstellt.
	\\
	Diese Datenbankstruktur war auch für die Entwicklung des Frontends von entscheidender Bedeutung, da nahezu jede Entität durch eine eigene Unterseite in der Anwendung repräsentiert wird.
	\\
	Ein vereinfachtes Entity-Relationship-Modell (ERM), das die Entitäten, Beziehungen und Kardinalitäten zeigt, befindet sich in Anhang A.9. In Anhang A.10 ist ein Darstellung der Datenbank mit ihren Tabellen zu finden.
	\subsection{Maßnahmen zur Qualitätssicherung}

	Ein zentrales Element der agilen Entwicklung bei der \acs{GSSD} ist die Durchführung regelmäßiger, technisch fokussierter Meetings. In diesen Sitzungen werden abgeschlossene Arbeitspakete präsentiert, gemeinsam mit dem Projektleiter getestet und gründlich analysiert. Die interdisziplinäre Zusammenarbeit ermöglicht eine fortlaufende Integration der erlangten Erkenntnisse in den Entwicklungsprozess, wodurch die Anwendungsleistung und Codequalität ständig optimiert werden.
	\\
	Die kontinuierliche Qualitätssicherung ermöglicht es, das anspruchsvolle Herausforderungen im Entwicklungsprozess effektiv zu bewältigen. Dies resultiert in einer erheblichen Zeitersparnis während der Entwicklung und schafft eine solide Grundlage für eine zukunftssichere und nachhaltige Weiterentwicklung der Anwendung im technischen Bereich.
	\\
	Um die Qualität des Projekts zu gewährleisten, wurden folgende Maßnahmen umgesetzt:
	\begin{enumerate}
	\item Unit- und Widget-Tests: Diese Testverfahren validieren die Richtigkeit der Implementierung und sorgen für eine zuverlässige Funktionalität der Anwendung.
	\item Regelmäßige Abstimmungen: Die Kommunikation mit dem Entwicklungsleiter dient dazu, Abweichungen frühzeitig zu identifizieren und entsprechende Korrekturmaßnahmen einzuleiten.
	\item GIT-Versionsverwaltung: Durch den Einsatz von GIT wird die Transparenz der Softwareentwicklung gewährleistet.
	\item Continuous Integration (CI): Mithilfe von Jenkins wird nach jedem Push-Vorgang eine automatische Prüfung der Software durchgeführt, um die Qualität ständig zu überwachen und sicherzustellen.
	\end{enumerate}

	\subsection{Deployment} Im Rahmen des Projekts wird die entwickelte Überwachungssoftware auf den Servern des Kunden implementiert und betriebsbereit gemacht. Dabei werden folgende Schritte unternommen:

	\begin{itemize}
		\item Installation der Software: Die entwickelte Anwendung wird auf den vom Kunden bereitgestellten Servern installiert. Dabei werden alle notwendigen Abhängigkeiten und Komponenten eingerichtet.
		\item Konfiguration von Systemeinstellungen: Systemeinstellungen und -parameter werden gemäß den Anforderungen und der Infrastruktur des Kunden angepasst.
		\item Integration mit anderen Anwendungen und Systemen: Die Überwachungssoftware wird mit bestehenden Anwendungen und Systemen des Kunden integriert, um eine nahtlose Zusammenarbeit und Informationsaustausch zu gewährleisten.
		\item Durchführung von Tests: Nachdem die Software erfolgreich installiert und konfiguriert wurde, werden Tests durchgeführt, um sicherzustellen, dass sie korrekt funktioniert und keine Fehler oder Probleme auftreten.
		\item Dokumentation: Die Installations- und Konfigurationsprozesse werden dokumentiert, um dem Kunden eine Referenz und Anleitung für zukünftige Anpassungen oder Updates zu bieten.
	\end{itemize}

	Das Deployment der Überwachungssoftware sollte sorgfältig geplant und durchgeführt werden, um mögliche Ausfallzeiten oder Probleme zu minimieren und einen reibungslosen Übergang zu gewährleisten. Hierzu wird ein entsprechender Plan erstellt, der die Aufgaben, Zuständigkeiten und Zeitpläne für das Deployment festlegt. Es werden auch Backup-Pläne und Wiederherstellungsmöglichkeiten eingerichtet, um Datenverluste oder Störungen zu vermeiden.
	\\
	Ein erfolgreicher Abschluss des Deployments stellt sicher, dass das Überwachungssystem effektiv arbeitet und den Kundenbedürfnissen entspricht.
	\subsection{Pflichtenheft}
	Zum Abschluss der Entwurfsphase wurde das Pflichtenheft erstellt. Die fachlichen Anforderungen, die gemeinsam mit mir im Abschnitt 3.5 (Lastenheft) erarbeitet wurden, bilden die Grundlage für das Pflichtenheft. Mithilfe des Pflichtenhefts kann am Ende überprüft werden, ob alle Anforderungen erfolgreich umgesetzt wurden und das Projektziel erreicht wurde. Das Pflichtenheft ist in Anhang A.19 auf Seite xviii zu finden.

	\section{Implementierungsphase}


	\subsection{Entwicklung des Dashboards}
	In der Entwicklungsphase des Dashboards lag der Fokus auf der Nutzung von wiederverwendbaren und verschachtelbaren Vue-Komponenten, um eine modulare und erweiterbare Benutzeroberfläche zu schaffen. Das Dashboard besteht aus diversen Komponenten, die spezifische Informationen, wie Benutzerstatistiken, Systemzustand oder Leistungskennzahlen, darstellen. Diese Komponenten wurden auf der Dashboard-Seite integriert, um einen umfassenden und aktuellen Einblick in die Leistung des Überwachungssystems zu ermöglichen.

	Die einzelnen Dashboard-Komponenten wurden so konzipiert, dass sie sowohl autonom als auch in Kombination miteinander funktionieren. Hierbei wurde die Flexibilität von Vue.js genutzt, um die Anordnung und Darstellung der Komponenten dynamisch an die Erfordernisse des Projekts anzupassen. Ein Beispiel hierfür ist die Implementierung von vuetify, einem Material Design Framework für Vue.js, das eine Vielzahl von vordefinierten Komponenten und Layouts bereitstellt.

	Dank der Anwendung asynchroner Funktionen und reaktiver Daten innerhalb der Dashboard-Komponenten werden die dargestellten Informationen kontinuierlich aktualisiert und auf dem neuesten Stand gehalten. Dies ermöglicht die Echtzeitüberwachung verschiedener Systemaspekte und trägt zur raschen Identifizierung von Problemen und Leistungseinbußen bei.

	Im Anhang sind beispielhafte Screenshots des entwickelten Dashboards enthalten, die verschiedene Komponenten und ihre Anordnung innerhalb der Benutzeroberfläche verdeutlichen. Diese Screenshots zeigen, wie das Dashboard eine strukturierte und benutzerfreundliche Visualisierung relevanter Informationen bietet und gleichzeitig flexibel genug ist, um an die spezifischen Anforderungen des Überwachungsprojekts angepasst zu werden.

	\subsection{Datenbank}
	Die verwendete Datenbank für das Projekt ist mariaDB. MariaDB ist ein weit verbreitetes, relationales Datenbankmanagementsystem, das sich durch seine Skalierbarkeit und Flexibilität auszeichnet. Es ist Open-Source und kann kostenfrei genutzt werden.\\
	Die Integration von mariaDB in das Backend ermöglicht die effiziente Verwaltung und Speicherung der gesammelten Daten aus den verschiedenen Anwendungen. Durch die Verwendung von mariaDB können auch komplexe Abfragen und Analysen der gespeicherten Daten durchgeführt werden, um wertvolle Erkenntnisse über die Leistung und Verfügbarkeit der überwachten Anwendungen zu gewinnen.\\
	Die Verbindung zwischen dem Backend und der mariaDB-Datenbank wird über entsprechende Treiber und Bibliotheken in Node.js hergestellt. Die Datenbankstruktur wird so gestaltet, dass sie leicht erweitert und angepasst werden kann, um neue Anwendungen oder Sensoren hinzuzufügen oder um zusätzliche Funktionen und Warnmeldungen zu unterstützen.\\
	Insgesamt bietet die Verwendung von mariaDB als Datenbanklösung für das Projekt eine robuste und skalierbare Grundlage, die den Anforderungen des Kunden gerecht wird und die Möglichkeit bietet, das Überwachungssystem im Laufe der Zeit weiterzuentwickeln und zu optimieren.\\


	\subsection{REST-API}
	In diesem Projekt wird eine REST-API verwendet, um den Datenaustausch zwischen den Sensoren und dem Backend zu ermöglichen. Die REST-API stellt eine standardisierte und leicht zu verwendende Schnittstelle bereit, um Daten zwischen diesen Systemkomponenten auszutauschen.
	\\
	Die Hauptmerkmale der in diesem Projekt verwendeten REST-API sind:
	\begin{enumerate}
	\item Zustandslosigkeit: Jede Anfrage von Client zu Server enthält alle notwendigen Informationen, sodass der Server den Kontext der Anfrage nicht speichern muss. Dies führt zu einer besseren Skalierbarkeit und Vereinfachung der Serverlogik.
	\item Cache-Fähigkeit: Die API-Antworten können gecacht werden, um die Leistung zu verbessern und die Last auf dem Server zu reduzieren.
	\item Client-Server-Architektur: Die REST-API trennt die Benutzeroberfläche (Frontend) von der Backend-Logik und der Datenverarbeitung. Dies erlaubt eine unabhängige Entwicklung und Verbesserung der einzelnen Komponenten.
	\item Einheitliche Schnittstelle: Die REST-API stellt eine einheitliche und konsistente Schnittstelle bereit, die die Interaktion zwischen den Komponenten vereinfacht.
	\end{enumerate}

	Im Kontext des Projekts werden die Sensoren die REST-API verwenden, um Daten an das Backend zu senden oder abzufragen. Das Backend empfängt die Daten von den Sensoren, analysiert und speichert sie in der Datenbank.

	Für die Kommunikation zwischen dem Backend und dem Frontend wird ein Websocket verwendet. Dies ermöglicht eine bidirektionale Kommunikation in Echtzeit zwischen den beiden Komponenten. Das Frontend verwendet den Websocket, um die Daten vom Backend abzurufen und sie in einer benutzerfreundlichen Weise darzustellen.

	Die Verwendung der REST-API für die Sensoren und des Websockets für die Kommunikation zwischen Backend und Frontend gewährleistet eine effiziente und zuverlässige Kommunikation, wodurch ein effektives Überwachungssystem für die Anwendungen auf den Servern
	des Kunden geschaffen wird.


	\subsection{Testen der Anwendung}
	Um sicherzustellen, dass die entwickelte Anwendung fehlerfrei funktionierte und die definierten Anforderungen erfüllte, war ein systematischer Testprozess erforderlich. Hierbei kamen verschiedene Testansätze und -ebenen zum Einsatz:

	\begin{enumerate}
	\item Unit-Tests
	\begin{itemize}
	\item Unit-Tests konzentrierten sich auf einzelne Codeeinheiten wie Funktionen oder Klassen. Sie gewährleisteten, dass diese korrekt arbeiteten und die erwarteten Ergebnisse lieferten.
	\item In der Regel wurden Unit-Tests mit Hilfe von Test-Frameworks wie JUnit Mocha (für JavaScript) erstellt und automatisiert ausgeführt.
	\end{itemize}
	\item Integrationstests
	\begin{itemize}
	\item Integrationstests überprüften die korrekte Interaktion zwischen verschiedenen Komponenten der Anwendung, wie zum Beispiel Datenbankzugriffe oder Kommunikation zwischen Backend und Frontend.
	\item Diese Tests wurden sowohl auf Code- als auch auf Systemebene durchgeführt, abhängig von den zu testenden Komponenten.
	\end{itemize}
	\item Systemtests
	\begin{itemize}
	\item Systemtests prüften die Anwendung in ihrer Gesamtheit, um sicherzustellen, dass alle Komponenten ordnungsgemäß zusammenspielten und die Anwendung wie beabsichtigt funktionierte.
	\item Diese Tests umfassten oft auch Last- und Performance-Tests, um die Leistungsfähigkeit und Stabilität der Anwendung unter verschiedenen Bedingungen zu überprüfen.
	\end{itemize}
	\item Akzeptanztests
	\begin{itemize}
	\item Akzeptanztests, auch bekannt als End-to-End-Tests oder Benutzertests, stellten sicher, dass die Anwendung den Anforderungen der Endbenutzer entsprach und alle definierten Use Cases abdeckte.
	\item Diese Tests wurden manuell durchgeführt oder mithilfe von Test-Tools wie Selenium automatisiert, um Benutzerinteraktionen mit der Anwendung zu simulieren.
	\end{itemize}
	\end{enumerate}

	Ein effektiver Testprozess umfasste sowohl manuelle als auch automatisierte Tests und folgte den Prinzipien des Test-Driven Development (TDD) oder Behavior-Driven Development (BDD), bei denen Tests als integraler Bestandteil des Entwicklungsprozesses betrachtet wurden. Regelmäßige Code-Reviews und die Verwendung von Continuous Integration (CI) und Continuous Deployment (CD) trugen ebenfalls zur Qualitätssicherung der Anwendung bei.

	\subsection{Sicherheitstests}
	Die Sicherheit des Überwachungssystems ist von großer Bedeutung, da es sensible Daten über die überwachten Anwendungen verarbeitet. Daher wurden Sicherheitstests durchgeführt, um potenzielle Schwachstellen und Angriffsvektoren zu identifizieren und entsprechende Gegenmaßnahmen zu implementieren. Dazu zählen unter anderem Tests zur Überprüfung der Authentifizierung, Autorisierung und Verschlüsselung.

	\section{Abnahmephase}

	\subsection{Code-Review}
	Code-Reviews waren ein wesentlicher Bestandteil des Entwicklungsprozesses, um die Codequalität zu gewährleisten und eine kontinuierliche Verbesserung des Codes zu fördern. Sie dienten dazu, potenzielle Fehler frühzeitig zu erkennen und Best Practices für die Codeentwicklung zu fördern.

	Während des Entwicklungsprozesses wurden Code-Reviews durchgeführt, um sicherzustellen, dass:

	\begin{enumerate}
	\item der Code den vereinbarten Programmierstandards und Richtlinien entsprach,
	\item der Code gut strukturiert, lesbar und wartbar war,
	\item der Code effizient und performant war,
	\item der Code frei von Sicherheitslücken und Anfälligkeiten war,
	\item der Code keine unbeabsichtigten Seiteneffekte oder Regressionen verursachte.
	\end{enumerate}

	Um Code-Reviews effektiv zu gestalten, wurden folgende Best Practices angewendet:

	\begin{enumerate}
	\item Die Verwendung von Pull Requests (PRs) in Git, um Änderungen am Code vor der Integration in den Hauptzweig zu überprüfen. Dies ermöglichte es den Entwicklern, Feedback zu geben und Probleme gemeinsam zu lösen.
	\item Die Einhaltung einer Checkliste für Code-Reviews, um sicherzustellen, dass alle wichtigen Aspekte des Codes überprüft wurden.
	\item Die Durchführung von regelmäßigen Code-Review, dabei geht es darum mögliche Verbesserungen zu realisieren.
	\end{enumerate}

	Durch das Implementieren dieser Praktiken wurden Code-Reviews zu einem wichtigen Instrument zur Qualitätssicherung und zur Verbesserung der Code.

	\subsection{Abnahme}
	Im Rahmen dieses Projekts fand keine separate Abnahmephase statt. Stattdessen wurden die Anwendung, ihre Benutzeroberfläche sowie die Interaktion zwischen Frontend und Backend kontinuierlich während des Entwicklungsprozesses in gemeinsamen Sprint-Reviews getestet und besprochen. Durch diesen Ansatz konnte eine interne Abnahme des Projekts bereits während der Entwicklungsphase sichergestellt werden.

	Trotzdem wurde eine formelle Testphase der Benutzeroberfläche, einschließlich JavaScript-Testfällen, durchgeführt. Aufgrund zeitlicher Verzögerungen kann diese Phase jedoch nicht ausführlich in dieser Projektdokumentation erklärt und berücksichtigt werden.


	\section{Dokumentation}
	Im Rahmen des Projektabschlusses für das spezifische Projekt, das sich auf die Entwicklung einer webbasierten Anwendung für die \acs{GSSD} konzentriert, wurden zwei zentrale Dokumentationen erstellt, um sowohl Endanwendern als auch Entwicklern eine umfassende Informationsquelle zur Verfügung zu stellen. Beide Dokumente wurden fachlich fundiert und ansprechend gestaltet, um ihren jeweiligen Zielgruppen den bestmöglichen Nutzen zu bieten. Im Folgenden werden die spezifischen Ziele und Inhalte dieser Dokumentationen in Bezug auf das Projekt erläutert:
	\subsection{Benutzerhandbuch Endanwender-Dokumentation}

	Das Benutzerhandbuch wurde für die Endanwender der \acs{GSSD} Anwendung konzipiert und bietet eine systematische Einführung in die Plattform, ihre Funktionen und Bedienung. Hierbei wurden ansprechende visuelle Darstellungen und präzise Anweisungen verwendet, um den Anwendern ein effizientes und angenehmes Nutzungserlebnis zu ermöglichen. Das Handbuch enthält detaillierte Schritt-für-Schritt-Anleitungen zur Verwendung der verschiedenen Module und Funktionen der Anwendung, wie z.B. Datenverwaltung, Berichterstellung und Zusammenarbeit zwischen verschiedenen Nutzergruppen. Zudem werden Best Practices und nützliche Tipps für den optimalen Einsatz der Software innerhalb der \acs{GSSD} bereitgestellt.

	\subsection{Entwicklerdokumentation}
	Die Entwicklerdokumentation richtet sich an Fachleute, die an der Weiterentwicklung, Wartung oder Integration der \acs{GSSD}-Anwendung beteiligt sind. Diese Dokumentation wurde sorgfältig erstellt, um ein tiefgehendes Verständnis der technischen Aspekte der Software zu vermitteln, einschließlich ihrer Architektur, des Codes, der verwendeten Technologien wie NodeJs, VueJs und MariaDB sowie der eingesetzten Bibliotheken und Frameworks. Um den Entwicklern eine strukturierte Orientierungshilfe zu bieten, wurden Best Practices und Richtlinien für die Projektstruktur, das Einrichten der Entwicklungsumgebung, das Durchführen von Tests und das Hinzufügen neuer Funktionen integriert. Die Entwicklerdokumentation soll somit eine solide Grundlage für eine effektive Zusammenarbeit und eine qualitativ hochwertige Weiterentwicklung des \acs{GSSD}-Projekts schaffen.


	\section{Fazit}
	Die vom mir entwickelte Überwachungssoftware erfüllt alle Anforderungen, die im Projektantrag festgelegt wurden. Sie bietet eine effiziente und zuverlässige Überwachung von Anwendungen auf den Servern des Kunden und behebt die Schwachstellen, die im bestehenden System identifiziert wurden. Die Anwendung ist skalierbar und anpassungsfähig an neue Anforderungen, sodass sie mit den Bedürfnissen des Kunden wachsen kann.

Während der Projektumsetzung traten einige Herausforderungen auf, wie die Anpassung des Zeitplans und die Ermittlung der Pflichten im Lastenheft. Diese wurden jedoch erfolgreich gemeistert, und die Erfahrungen daraus können für zukünftige Projekte genutzt werden. Die Zusammenarbeit im die Kommunikation mit dem Kunden waren ebenfalls wichtige Aspekte, die während des Projekts berücksichtigt und erfolgreich bewältigt wurden.
	\subsection{Soll-\/Ist-Vergleich}
	Nach Abschluss des Projekts kann festgestellt werden, dass alle im Projektantrag genannten Anforderungen erfolgreich umgesetzt wurden. Jedoch musste der Zeitplan im Verlauf des Projekts angepasst werden, da die Umsetzung des Konzepts mehr Zeit in Anspruch nahm als ursprünglich vermutet. Die Analyse des Lastenhefts und die Ermittlung der Pflichten waren aufgrund der hohen Anzahl an einzelnen Komponenten und des komplexen Prozesses der Rechtevergabe zeitintensiver als erwartet. Die Projektplanungsphase wurde aufgrund der fehlenden Erfahrungen im Vorfeld als aufwendiger eingeschätzt. Die zeitliche Schätzung der weiteren Phasen erwies sich
	insgesamt als zutreffend. Etwa 9 Stunden Arbeitszeit konnten keiner Projektphase direkt zugeordnet werden, da sie für die Überprüfung der Rahmenbedingungen wie das Layout und die Erstellung erforderlicher Anlagen verwendet wurden.

	\subsection{Lessons Learned} Während des Projekts wurden verschiedene Erfahrungen gesammelt und wichtige Erkenntnisse gewonnen, die in zukünftigen Projekten genutzt werden können, um den Erfolg und die Effizienz der Projektarbeit zu verbessern. Einige der wichtigsten Lessons Learned sind:

	\begin{itemize} \item \textbf{Realistische Zeitplanung:} Die Anpassung des Zeitplans während der Projektumsetzung zeigt, wie wichtig es ist, genügend Zeit für die verschiedenen Phasen und Aufgaben einzuplanen. Dabei sollte auch ausreichend Pufferzeit für unvorhergesehene Probleme oder Verzögerungen berücksichtigt werden.

	\item \textbf{Kommunikation und Zusammenarbeit:} Obwohl das Projekt allein durchgeführt wurde, war eine klare Kommunikation mit den Stakeholdern, insbesondere dem Kunden, entscheidend für den Projekterfolg. Regelmäßige Updates und Feedback-Runden stellen sicher, dass alle Anforderungen korrekt verstanden und umgesetzt werden und ermöglichen eine kontinuierliche Verbesserung während des Entwicklungsprozesses.

	\item \textbf{Umfassende Analyse und Dokumentation:} Eine gründliche Analyse der bestehenden Systeme und der Anforderungen des Kunden ist entscheidend für die erfolgreiche Umsetzung eines Projekts. Eine ausführliche und gut strukturierte Dokumentation hilft dabei, den Projektverlauf nachzuvollziehen und eventuelle Probleme oder Verbesserungsmöglichkeiten zu identifizieren.

	\item \textbf{Fehlerbehebung und Tests:} Eine sorgfältige Fehlerbehebung und umfassende Tests während des gesamten Entwicklungsprozesses stellen sicher, dass die entwickelte Lösung den Anforderungen entspricht und funktioniert, wie sie soll. Frühzeitiges Erkennen und Beheben von Fehlern minimiert das Risiko von Problemen in späteren Projektphasen.

	\item \textbf{Anpassungsfähigkeit und Lernbereitschaft:} Die Fähigkeit, sich an Veränderungen oder neue Anforderungen während des Projekts anzupassen, ist entscheidend für den Projekterfolg. Die Bereitschaft, neue Technologien oder Ansätze zu erlernen und anzuwenden, kann dazu beitragen, die Projekteffizienz zu steigern und die Qualität der entwickelten Lösung zu verbessern.

	\item \textbf{Planung für Skalierbarkeit:} Bei der Entwicklung der Überwachungssoftware war es wichtig, die Skalierbarkeit der Lösung zu berücksichtigen, um zukünftige Erweiterungen oder Anpassungen an neue Anforderungen zu ermöglichen. Das Einplanen von Skalierbarkeit von Anfang an reduziert die Wahrscheinlichkeit, dass umfangreiche Änderungen in späteren Projektphasen oder nach der Fertigstellung erforderlich sind.


	\end{itemize}

	Die gewonnenen Erkenntnisse und Lessons Learned können dazu beitragen, zukünftige Projekte besser zu planen, durchzuführen und abzuschließen, was zu höherer Effizienz, erfolgreichen Projektergebnissen und



\end{flushleft}
